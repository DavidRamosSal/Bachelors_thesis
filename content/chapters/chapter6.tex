\chapter*{Conclusiones}
\addcontentsline{toc}{chapter}{Conclusiones}
%El comportamiento de la materia en las estrellas de neutrones 
%Al día de hoy nuestra comprensión de las leyes de la física a densidades altas sigue siendo pobre. 
%La EOS de la materia en el interior de las estrellas de neutrones sigue siendo al día de hoy un misterio. Además de su importancia práctica en simulaciones de diferentes eventos astrofísicos,
\noindent Más allá de su utilidad práctica en simulaciones de supernovas y otros eventos astronómicos, la importancia de determinar la EOS de la materia al interior de las estrellas de neutrones reside en que nos permite entender el comportamiento de las leyes de la física a densidades que no son realizables experimentalmente en la tierra. La existencia de múltiples candidatos sólo refleja nuestra inhabilidad para aplicar nuestro conocimiento de física nuclear en condiciones tan extremas como las encontradas en las estrellas de neutrones. El objetivo de este trabajo fue aplicar criterios de aceptabilidad física a los modelos de estrellas de neutrones estáticas obtenidos con diferentes EOSs con el fin de identificar candidatos que produzcan modelos que no sean realistas. 

Tras una breve discusión sobre los aspectos generales de las estrellas de neutrones, se presentó una deducción de las ecuaciones de estructura estelar tanto en la teoría Newtoniana como en la relativista y se explicó cómo brindando una EOS es posible construir modelos estelares. Las condiciones de aceptabilidad fueron presentadas y aplicadas a un conjunto de 37 EOSs que varían ampliamente en composición y rigidez. En lo que concierne a las condiciones C1-10 los resultados obtenidos en este trabajo habían sido presentados en los artículos originales y parcialmente en otros trabajos \cite{Ozel2016,Read2009}, excepto para la condición C4: el problema encontrado al aplicar los criterios existentes para restringir el valor del índice adiabático al interior de la estrella no ha sido reportado. 

Los resultados obtenidos para la condición C11 fueron inesperados: la corteza interior de los modelos obtenidos con todas las EOSs consideradas es inestable ante movimientos convectivos adiabáticos. La posibilidad de que la corteza interna presente inestabilidades convectivas no ha sido reportada antes y debido a la complejidad de los modelos físicos desarrollados para describir fenómenos como la superfluidez y cambios de fase que ocurren esta región, es difícil identificar posibles causas. Este resultado puede ser importante dado que se cree que existe convección de superfluido de esta región hacia otras \cite{Haensel2007}. 

Respecto a la confiabilidad de los resultados obtenidos cabe resaltar que la rutina numérica reproduce diferentes soluciones exactas de fluido perfecto con gran precisión y que los problemas de usar diferencias finitas para calcular derivadas numéricas fueron superados con la ayuda de un spline con un factor de suavizamiento. Así mismo, los resultados de la masas máximas y los respectivos radios para cada EOS concuerdan con lo presentado en otros trabajos.

Una de las inquietudes que puede surgir al aplicar el criterio C11 a modelos de estrellas de neutrones es que éste es Newtoniano en esencia (es una consecuencia directa del principio de Arquímedes) y su validez en el contexto de la relatividad general no está asegurada. Sin embargo, como ha sido argumentado para criterios de estabilidad ante convección Newtonianos más generales, a diferencia de las pulsaciones que son fenómenos globales, la inestabilidad ante convección es un fenómeno local que está gobernado enteramente por los valores locales de las variables termodinámicas \cite{Thorne1966}.    

La metodología de este trabajo es análoga a la presentada en \cite{Delgaty1998} para soluciones exactas y, en vista del resultado obtenido para la estabilidad convectiva, complementa el uso de observaciones para restringir modelos de EOSs de una manera fundamental, puesto que permite identificar partes específicas del conocimiento actual en las que es posible no se haya explorado todo el rango de posibilidades físicas.

Finalmente, se espera extender el trabajo realizado para incluir más EOSs, incluyendo modernas EOSs de "alta precisión" que permitirían reducir los errores numéricos inherentes a la interpolación y analizar mejor la estabilidad de la corteza interior.