\chapter{Solución numérica de las ecuaciones de TOV}\label{NumSol}
En este apéndice se discutirá el esquema numérico utilizado para solucionar las ecuaciones de TOV y se verificará la confiabilidad de éste comparando sus resultados con algunas de las soluciones interiores exactas conocidas.

La implementación en Python de la información presente en este apéndice se encuentra en el notebook de Jupyter \href{https://nbviewer.jupyter.org/github/DavidRamosSal/stellar_structure/blob/master/Static_structure_manual.ipynb}{Static Structure Manual}.

\section{Escalamiento del sistema de ecuaciones}
Las ecuaciones de TOV en unidades gravitacionales son
\begin{align}
    \frac{dm}{dr}=&4\pi \rho r^2 ,\\
    \frac{dP}{dr}=&-(\rho+P)\frac{m+4\pi r^3 P}{r(r-2m)} , \\
    \frac{d\nu}{dr}=& \frac{m+4\pi r^3 P}{r(r-2m)} =  -\frac{1}{\rho+P}\frac{dP}{dr},
\end{align}
sujetas a una ligadura dada por la ecuación de estado $P=P(\rho)$ y valores iniciales $\rho(r=0)=\rho_c$, $m(r=0)=0$ y $\nu(r=0)=\text{cte}$.

Debido a que los valores de las cantidades físicas cambian en varios órdenes de magnitud a lo largo de la extensión de la estrella, es necesario/preferible/útil re-escalar todas las variables dimensionales usando una constante con las mismas dimensiones conocida como la 'escala', esto con el doble propósito de convertir las variables en variables adimensionales y hacer su valor alrededor de la unidad \cite{Langtangen2016ScalingEquations}.

Re-escalando las variables como
\begin{equation}
    \rho=\rho_* \bar{\rho} \quad ; \quad P=P_* \bar{P} \quad ; \quad m=m_*\bar{m} \quad ; \quad r=r_*\bar{r},
\end{equation}
el sistema de ecuaciones se convierte en
\begin{align}
\frac{d\bar{m}}{\bar{dr}}=&\left( \frac{\rho_* r_*^3}{m_*} \right) 4\pi \bar{\rho} \bar{r}^2 ,\\
\frac{d\bar{P}}{d\bar{r}}=&-\left( \frac{m_*}{r_* \rho_*} \right)\left(\frac{\rho_*}{P_*}\bar{\rho}+\bar{P}\right)\frac{\bar{m}+\left( \frac{r_*^3 P_*}{m_*} \right)4\pi \bar{r}^3 \bar{P}}{\bar{r}\left(\bar{r}-\frac{m_*}{r_*}2\bar{m}\right)} , \\
 \frac{d\nu}{d\bar{r}}=&-\frac{1}{\left(\bar{P}+\frac{\rho_*}{P_*}\bar{\rho}\right)}\frac{d\bar{P}}{d\bar{r}}.
\end{align}
Escogiendo la escala de $m$, $P$ y $\nu$ de modo que el sistema de ecuaciones adimensionales mantenga la forma del sistema original se obtienen las relaciones
\begin{equation}
    P_*=\rho_*,\quad r_*=\frac{1}{\sqrt{\rho_*}} ,\quad m_*=r_* ,
\end{equation}
así que se pueden escribir todas las constantes de escalamiento en términos de una sola.
Escogiendo $\rho_*$ como la constante independiente, el último paso consiste en usar un valor que refleje la escala de densidades del problema. A partir de la masa del neutrón $m_n$, se puede construir una densidad $\rho_* = \frac{m_{n}^{4}c^{3}}{8 \pi^2 \hbar^3}\sim 2\times 10^{15} \text{g/cm}^{3}$ como un valor característico de las densidades centrales de las estrellas de neutrones.

El sistema re-escalado a solucionar será
\begin{align}\label{adimensional}
      \frac{d\bar{m}}{d\bar{r}}=&4\pi \bar{\rho} \bar{r}^2 , \nonumber \\
    \frac{d\bar{P}}{d\bar{r}}=&-(\bar{\rho}+\bar{P})\frac{\bar{m}+4\pi \bar{r}^3 \bar{P}}{\bar{r}(\bar{r}-2\bar{m})} , \\
    \frac{d\bar{\nu}}{d\bar{r}}=& \frac{\bar{m}+4\pi \bar{r}^3 \bar{P}}{\bar{r}(\bar{r}-2\bar{m})} =  -\frac{1}{\bar{\rho}+\bar{P}}\frac{d\bar{P}}{d\bar{r}},\nonumber
\end{align}
con la ecuación de estado expresada en términos de las variables adimensionales $\bar{P}=\bar{P}(\bar{\rho})$ y los valores iniciales $\bar{\rho}(\bar{r}=0)=\rho_c / \rho_{*}$, $\bar{m}(\bar{r}=0)=0$ y $\bar{\nu}(\bar{r}=0)=\text{cte}$.

\section{Integración del sistema}

\TODO{Aproximación de las ecuaciones para $r=0$.}

La integración numérica del sistema de ecuaciones diferenciales ordinarias \eqref{adimensional} se realizó usando el método de Runge-Kutta de cuarto orden, \TODO{¿Explicar el método?} donde el paso de integración se escogió como una fracción $\delta$ de una escala de distancias definida a partir de los gradientes de presión y masa \cite{Baym1971TheModels}: 
\begin{equation}
    \Delta{r} = \delta \left( \frac { 1 } { m } \frac { \mathop{dm} } { \mathop{dr}  } - \frac { 1 } { P } \frac { \mathop{dP}  } { \mathop{dr} } \right) ^ { - 1 }.
\end{equation}

Haciendo uso de este paso adaptativo se evade la necesidad de tener un estimado del radio de la estrella al determinar un valor sensible para el paso fijo, pues $\Delta$ disminuye cuando $m$ o $P$ cambian rápidamente.

Debido a que las ecuaciones de estado consideradas se encuentran tabuladas solo para ciertos valores de $\rho$ y $P$, esta debe ser interpolada para los valores intermedios requeridos en cada paso de integración. Esto se realizó interpolando linealmente $\log{\rho}$ como una función de $\log{P}$. \TODO{Entender bien por qué es distinto a simplemente interpolar $P$ y $\rho$}

\section{Comparando con soluciones exactas}

Para verificar que los resultados obtenidos usando la rutina numérica creada son confiables, estos pueden ser comparados con algunas de las soluciones exactas a las ecuaciones de TOV que se obtienen al imponer ciertas relaciones entre las variables físicas o una ecuación de estado lo suficientemente simple. 

A continuación se presentará solamente la comparación con la solución Tolman VII \TODO{Cita al paper de Tolman?}, pues es una de las soluciones exactas de mayor interés por su posibilidad de modelar configuraciones estelares realistas \TODO{Cita de aplicaciones y estabilidad}. Comparaciones con otras soluciones exactas pueden ser encontradas en el Notebook de Jupyter.    

\section{Derivadas numéricas de las variables físicas}\label{NumDer}  