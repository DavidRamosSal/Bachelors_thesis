\chapter{Curvatura en un espacio-tiempo estático con simetría esférica}

La métrica de un espacio-tiempo estático con simetría esférica está dada por
\begin{equation}
    \bm{g}=-e^{2 a(r)} \dd{t} \otimes \dd{t}+e^{2 b(r)} \dd{r} \otimes \dd{r} +r^{2}\left(\dd{\theta} \otimes \dd{\theta}+\operatorname{sen}^{2} \theta \dd{\varphi} \otimes \dd{\varphi}\right),
\end{equation}
el tensor de curvatura de Riemann correspondiente a esta métrica puede ser calculado de manera simple haciendo uso del cálculo de Cartan. 

En el cálculo de Cartan se hace uso del hecho de que es posible escoger como base del espacio cotangente $T^{*}_p$ en un punto $p$ de la variedad, una base $\omega^{\alpha}$ diferente a la base coordenada $\dd{x^\alpha}$. Esta base se escoge de modo tal que la métrica pueda ser escrita como 
\begin{equation}
    \bm{g}=\tensor{\eta}{_\alpha_\beta} \omega^{\alpha} \otimes \omega^{\beta},
\end{equation}
donde $\tensor{\eta}{_\alpha_\beta}$ es la métrica de Minkowski. Este tipo de bases se conocen como bases ortonormales y para ellas las 1-formas de conexión $\tensor{\Gamma}{^\alpha_\beta}$ cumplen la condición
\begin{equation}
    \Gamma_{\alpha \beta}+\Gamma_{\alpha \beta}=0. \label{skew}
\end{equation}
Con lo anterior en mente, se define la tétrada de 1-formas base como
%\begin{subequations}
\begin{equation}
    \omega^0=\,e^{a}\dd{t}, \quad
    \omega^1=\,e^{b}\dd{r}, \quad
    \omega^2=\,r\dd{\theta}, \quad
    \omega^3=\,r\sen{\theta}\dd{\varphi}.
    \label{tetrada}
\end{equation}
%\end{subequations}
Usando las ecuaciones de estructura de Cartan para un espacio-tiempo sin torsión\TODO{Citar a Chandrasekar o Straumann}
\begin{subequations}
\begin{align}
    \dd{\omega^\alpha} =& -\tensor{\Gamma}{^\alpha_ \mu} \wedge \omega^{\mu}\,, \label{1SE}\\
    \tensor{\Omega}{^\alpha_\beta} =& \dd{ \tensor{\Gamma}{^\alpha_\beta}} + \tensor{\Gamma}{^\alpha_\mu}  \wedge \tensor{\Gamma}{^\mu_\beta}\,, \label{2SE}
\end{align}
\end{subequations}
se pueden obtener las las 2-formas de curvatura $\tensor{\Omega}{^\alpha_\beta}$ y a partir de estas, las componentes del tensor de Riemann pues por definición 
\begin{equation}
    \tensor{\Omega}{^\alpha_\beta}=\frac{1}{2}\tensor{R}{^\alpha_\beta_\mu_\nu}\omega^\mu \wedge \omega^\nu.
\end{equation}
Comenzando por el cálculo de las 1-formas de conexión, se hallan las derivadas exteriores de la tétrada \eqref{tetrada}
    \begin{align}
    \begin{split}
        \dd{\omega^0}=&\dd{e^a} \wedge \dd{t}+e^a\cancelto{0}{\dd{(\dd{t})}} = a^{\prime} e^a \dd{r}\wedge\dd{t} = -a^{\prime} e^{-b} \omega^0 \wedge \omega^1, \\
        \dd{\omega^1}=&\dd{e^b}\wedge\dd{r}+e^b\cancelto{0}{\dd{(\dd{r})}} = b^{\prime}e^b\dd{r}\wedge\dd{r}=0, \\
        \dd{\omega^2}=&\dd{r}\wedge\dd{\theta} + r\cancelto{0}{\dd{(\dd{\theta})}} = -\frac{e^{-b}}{r}\omega^2 \wedge\omega^1, \\
        \dd{\omega^3}=&\dd{(rsen\theta)}\wedge\dd{\varphi} + r\sen{\theta}\cancelto{0}{\dd{(\dd{\varphi})}} = \sen{\theta}\dd{r}\wedge\dd{\varphi} + r\cos{\theta}\dd{\theta}\wedge\dd{\varphi} \\
        & \phantom{{}\dd{(rsen\theta)}\wedge\dd{\varphi} + r\sen{\theta}\cancelto{0}{\dd{(\dd{\varphi})}}} = -\frac{e^{-b}}{r} \omega^3 \wedge \omega^1 - \frac{\cot{\theta}}{r} \omega^3 \wedge \omega^2,
    \end{split}
    \end{align}
reemplazando en la primera ecuación de estructura \eqref{1SE}
    \begin{align*}
        \tensor{\Gamma}{^0_0}\wedge\omega^0 + \tensor{\Gamma}{^0_1}\wedge\omega^1+\tensor{\Gamma}{^0_2}\wedge\omega^2+\tensor{\Gamma}{^0_3}\wedge\omega^3 =& \,a^{\prime} e^{-b} \omega^0 \wedge \omega^1,  \\
        \tensor{\Gamma}{^1_0}\wedge\omega^0 + \tensor{\Gamma}{^1_1}\wedge\omega^1+\tensor{\Gamma}{^1_2}\wedge\omega^2+\tensor{\Gamma}{^1_3}\wedge\omega^3 =&\, 0, \\
        \tensor{\Gamma}{^2_0}\wedge\omega^0 + \tensor{\Gamma}{^2_1}\wedge\omega^1+\tensor{\Gamma}{^2_2}\wedge\omega^2+\tensor{\Gamma}{^2_3}\wedge\omega^3 =& \frac{e^{-b}}{r}\omega^2 \wedge\omega^1, \\
        \tensor{\Gamma}{^0_0}\wedge\omega^0 + \tensor{\Gamma}{^3_1}\wedge\omega^1+\tensor{\Gamma}{^3_2}\wedge\omega^2+\tensor{\Gamma}{^3_3}\wedge\omega^3 =& \frac{e^{-b}}{r} \omega^3 \wedge \omega^1 + \frac{\cot{\theta}}{r} \omega^3 \wedge \omega^2,
    \end{align*}    
de donde se pueden identificar las 1-formas de conexión no nulas
\begin{subequations}
    \begin{align}
        \tensor{\Gamma}{^0_1} =& \,\tensor{\Gamma}{^1_0} = a^{\prime} e^{-b} \omega^0, \label{1fa} \\
        \tensor{\Gamma}{^2_1} =& -\tensor{\Gamma}{^1_2} = \frac{e^{-b}}{r}\omega^2, \label{1fb} \\
        \tensor{\Gamma}{^3_1} =& -\tensor{\Gamma}{^1_3} = \frac{e^{-b}}{r}\omega^3, \label{1fc} \\
        \tensor{\Gamma}{^3_2} =& -\tensor{\Gamma}{^2_3} = \frac{\cot{\theta}}{r} \omega^3, \label{1fd}
    \end{align}
\end{subequations}
donde las relaciones entre las 1-formas se siguen de la relación \eqref{skew}.

Hallando las derivadas exteriores de \eqref{1fa}--\eqref{1fd} 
\begingroup
\allowdisplaybreaks
\begin{subequations}
    \begin{align}
        \dd{\tensor{\Gamma}{^0_1}} =& \dd{(a^\prime e^{-b})}\wedge\omega^0 + a^\prime e^{-b}\dd{\omega^0} \nonumber \\
        =& e^{-b}\dd{(a^\prime)}\wedge\omega^0 + a^\prime \dd{(e^{-b})}\wedge\omega^0-a^{\prime 2} e^{-2b}\omega^0\wedge\omega^1 \nonumber \\
        =& a^{\prime\prime} e^{-b} \dd{r}\wedge\omega^0 -a^{\prime}b^{\prime}e^{-b}\dd{r}\wedge\omega^0-a^{\prime 2}e^{-2b}\omega^0\wedge\omega^1 \nonumber \\
        =& e^{-2b}(a^{\prime\prime}+a^{\prime 2}-a^\prime b^\prime) \omega^1 \wedge \omega^0, \\
        \dd{\tensor{\Gamma}{^2_1}} =& \dd{\left(\frac{e^{-b}}{r}\right)}\wedge\omega^2 + \frac{e^{-b}}{r}\dd{\omega^2} \nonumber \\
        =& \frac{1}{r}\dd{e^{-b}}\wedge\omega^2 + e^{-b} \dd{\left(\frac{1}{r}\right)}\wedge\omega^2 + \frac{e^{-2b}}{r^2}\omega^1 \wedge \omega^2 \nonumber \\
        =& -\frac{b^\prime}{r}e^{-b}\dd{r}\wedge\omega^2 -\cancel{\frac{e^{-b}}{r^2}\dd{r}\wedge\omega^2} +\cancel{\frac{e^{-2b}}{r^2}\omega^1\wedge\omega^2} \nonumber \\
        =& - \frac{b^\prime e^{-2b}}{r} \omega^1 \wedge \omega^2, \\
        \dd{\tensor{\Gamma}{^2_1}} =& \dd{\left(\frac{e^{-b}}{r}\right)}\wedge\omega^3 + \frac{e^{-b}}{r}\dd{\omega^3} \nonumber \\
        =& \frac{1}{r}\dd{\left(e^{-b}\right)}\wedge\omega^3 + e^{-b}\dd{\left(\frac{1}{r}\right)}\wedge\omega^3+\frac{e^{-2b}}{r^2}\omega^1\wedge\omega^3 + \frac{e^-b}{r^2}\cot{\theta}\omega^2\wedge\omega^3 \nonumber \\
        =& -\frac{b^\prime e^{-2b}}{r}\dd{r}\wedge\omega^3-\cancel{\frac{e^{-b}}{r^2}\dd{r}\wedge\omega^3} + \cancel{\frac{e^{-2b}}{r^2}\omega^1\wedge\omega^3}+\frac{e^{-b}}{r^2}\cot{\theta}\omega^2\wedge\omega^3 \nonumber \\
        =& -\frac{b^\prime e^{-2b}}{r}\omega^1\wedge\omega^3 + \frac{e^{-b}\cot{\theta}}{r^2}\omega^2\wedge\omega^3, \\
        \dd{\tensor{\Gamma}{^3_2}}=& \dd{\left( \frac{\cot{\theta}}{r} \right)}\wedge\omega^3 + \frac{\cot{\theta}}{r}\dd{\omega^3} \nonumber \\ 
        =& \frac{1}{r}\dd{\qty(\cot{\theta})}+\cot{\theta}\dd{\qty(\frac{1}{r})}\wedge\omega^3 \nonumber \\
        =& -\frac{\csc^2{\theta}}{r}\dd{\theta}\wedge\omega^3 -\cancel{ \frac{\cot{\theta}}{r^2}\dd{r}\wedge\omega^3}+\cancel{\frac{\cot{\theta}e^{-b}}{r^2}\omega^1\wedge\omega^3}+\frac{\cot^2{\theta}}{r^2}\omega^2\wedge\omega^3 \nonumber \\
        =& \frac{1}{r^2}\cancelto{-1}{(\cot^2{\theta}-\csc^2{\theta})}\omega^2\wedge\omega^3 \nonumber \\
        =& -\frac{1}{r^2}\omega^2\wedge\omega^3.
    \end{align}
\end{subequations}
\endgroup
Reemplazando las 1-formas de conexión y sus derivadas exteriores en la segunda ecuación de estructura \eqref{2SE} se obtienen las 2-formas de curvatura $\tensor{\Omega}{^\alpha_\beta}$
\begingroup
\allowdisplaybreaks
\begin{subequations}
    \begin{align}
        \tensor{\Omega}{^0_1}=&\dd{\tensor{\Gamma}{^0_1}}+\cancelto{0}{\tensor{\Gamma}{^0_0}}\wedge\tensor{\Gamma}{^0_1}+\tensor{\Gamma}{^0_1}\wedge\cancelto{0}{\tensor{\Gamma}{^1_1}}+\cancelto{0}{\tensor{\Gamma}{^0_2}}\wedge\tensor{\Gamma}{^2_1}+\cancelto{0}{\tensor{\Gamma}{^0_3}}\wedge\tensor{\Gamma}{^3_1} \nonumber \\
        =& -e^{-2b}(a^{\prime\prime}+a^{\prime 2}-a^\prime b^\prime)\omega^0\wedge\omega^1 = \frac{1}{2}\tensor{R}{^0_1_\alpha_\beta}\omega^\alpha\wedge\omega^\beta, \\
        \tensor{\Omega}{^0_2}=&\dd{\cancelto{0}{\tensor{\Gamma}{^0_2}}}+\cancelto{0}{\tensor{\Gamma}{^0_0}}\wedge\tensor{\Gamma}{^0_2}+\tensor{\Gamma}{^0_1}\wedge\tensor{\Gamma}{^1_2}+\cancelto{0}{\tensor{\Gamma}{^0_2}}\wedge\cancelto{0}{\tensor{\Gamma}{^2_2}}+\cancelto{0}{\tensor{\Gamma}{^0_3}}\wedge\tensor{\Gamma}{^3_2} \nonumber \\
        =& -\frac{a^\prime e^{-2b}}{r}\omega^0\wedge\omega^2= \frac{1}{2}\tensor{R}{^0_2_\alpha_\beta}\omega^\alpha\wedge\omega^\beta, \\
        \tensor{\Omega}{^0_3}=&\dd{\cancelto{0}{\tensor{\Gamma}{^0_3}}}+\cancelto{0}{\tensor{\Gamma}{^0_0}}\wedge\cancelto{0}{\tensor{\Gamma}{^0_3}}+\tensor{\Gamma}{^0_1}\wedge\tensor{\Gamma}{^1_3}+\cancelto{0}{\tensor{\Gamma}{^0_2}}\wedge\tensor{\Gamma}{^2_3}+\cancelto{0}{\tensor{\Gamma}{^0_3}}\wedge\cancelto{0}{\tensor{\Gamma}{^3_3}} \nonumber \\
        =& -\frac{a^\prime e^{-2b}}{r}\omega^0\wedge\omega^3= \frac{1}{2}\tensor{R}{^0_3_\alpha_\beta}\omega^\alpha\wedge\omega^\beta, \\
        \tensor{\Omega}{^1_2}=&\dd{\tensor{\Gamma}{^1_2}}+\tensor{\Gamma}{^1_0}\wedge\cancelto{0}{\tensor{\Gamma}{^0_2}}+\cancelto{0}{\tensor{\Gamma}{^1_1}}\wedge\tensor{\Gamma}{^1_2}+\tensor{\Gamma}{^1_2}\wedge\cancelto{0}{\tensor{\Gamma}{^2_2}}+\tensor{\Gamma}{^1_3}\wedge\tensor{\Gamma}{^3_2} \nonumber \\
        =& \frac{e^{-2b}}{r}\omega^1\wedge\omega^2-\frac{e^{-b}\cot{\theta}}{r^2}\cancelto{0}{\omega^3\wedge\omega^3} \nonumber \\
        =& \frac{b^\prime e^{-2b}}{r}\omega^1\wedge\omega^2 = \frac{1}{2}\tensor{R}{^1_2_\alpha_\beta}\omega^\alpha\wedge\omega^\beta, \\
        \tensor{\Omega}{^1_3}=&\dd{\tensor{\Gamma}{^1_3}}+\tensor{\Gamma}{^1_0}\wedge\cancelto{0}{\tensor{\Gamma}{^0_3}}+\cancelto{0}{\tensor{\Gamma}{^1_1}}\wedge\tensor{\Gamma}{^1_3}+\tensor{\Gamma}{^1_2}\wedge\tensor{\Gamma}{^2_3}+\tensor{\Gamma}{^1_3}\wedge\cancelto{0}{\tensor{\Gamma}{^3_3}} \nonumber \\
        =& \frac{b^\prime e^{-2b}}{r}\omega^1\wedge\omega^3-\cancel{\frac{e^{-b}\cot{\theta}}{r^2}\omega^2\wedge\omega^3}+\cancel{\frac{e^{-b}\cot{\theta}}{r^2}\omega^2\wedge\omega^3} \nonumber \\
        =& \frac{b^\prime e^{-2b}}{r}\omega^1\wedge\omega^3 = \frac{1}{2}\tensor{R}{^1_3_\alpha_\beta}\omega^\alpha\wedge\omega^\beta, \\
        \tensor{\Omega}{^2_3}=&\dd{\tensor{\Gamma}{^2_3}}+\cancelto{0}{\tensor{\Gamma}{^2_0}}\wedge\cancelto{0}{\tensor{\Gamma}{^0_3}}+\tensor{\Gamma}{^2_1}\wedge\tensor{\Gamma}{^1_3}+\cancelto{0}{\tensor{\Gamma}{^2_2}}\wedge\tensor{\Gamma}{^2_3}+\tensor{\Gamma}{^2_3}\wedge\cancelto{0}{\tensor{\Gamma}{^3_3}} \nonumber \\
        =& \frac{1}{r^2}\omega^2\wedge\omega^3-\frac{e^{-2b}}{r^2}\omega^2\wedge\omega^3 \nonumber \\
        =& \frac{1-e^{-2b}}{r^2}\omega^2\wedge\omega^3 = \frac{1}{2}\tensor{R}{^2_3_\alpha_\beta}\omega^\alpha\wedge\omega^\beta,
    \end{align}
\end{subequations}
\endgroup
y por inspección se obtienen las componentes independientes del tensor de Riemann

\begin{align}
    \begin{split}
        \tensor{R}{^0_1_0_1} &= -e^{-2b}(a^{\prime\prime}+a^{\prime 2}-a^\prime b^\prime), \\
        \tensor{R}{^0_2_0_2} &= -\frac{a^\prime e^{-2b}}{r}, \\ 
        \tensor{R}{^0_3_0_3} &= -\frac{a^\prime e^{-2b}}{r}, \\
        \tensor{R}{^1_2_1_2} &= \frac{b^\prime e^{-2b}}{r}, \\
        \tensor{R}{^1_3_1_3} &= \frac{b^\prime e^{-2b}}{r}, \\
        \tensor{R}{^2_3_2_3} &= \frac{1-e^{-2b}}{r^2}.
    \end{split}
\end{align}
Usando la antisimetría en el primer y segundo par de índices
\begin{equation}
    \tensor{\eta}{_\mu_\alpha}\tensor{R}{^\mu_\beta_\gamma_\delta}=-\tensor{\eta}{_\mu_\beta}\tensor{R}{^\mu_\alpha_\gamma_\delta}\quad \text{y} \quad \tensor{R}{^\alpha_\beta_\gamma_\delta}=\tensor{R}{^\alpha_\beta_\delta_\gamma},
\end{equation}
se obtienen las componentes restantes
\begin{equation}
    \begin{split}
    \tensor{R}{^0_1_0_1} &= \tensor{R}{^1_0_0_1} = -\tensor{R}{^0_1_1_0} = -\tensor{R}{^1_0_1_0}, \\
    \tensor{R}{^0_2_0_2} &=  \tensor{R}{^2_0_0_2} = - \tensor{R}{^0_2_2_0} = - \tensor{R}{^2_0_2_0}, \\
    \tensor{R}{^0_3_0_3} &= \tensor{R}{^3_0_0_3} = -\tensor{R}{^0_3_3_0}=-\tensor{R}{^3_0_3_0},
    \end{split}
    \qquad
    \begin{split}
    \tensor{R}{^1_2_1_2} &= \tensor{R}{^2_1_2_1} = -\tensor{R}{^1_2_2_1}=-\tensor{R}{^2_1_1_2}, \\
    \tensor{R}{^1_3_1_3} &= \tensor{R}{^3_1_3_1} = -\tensor{R}{^1_3_3_1} = -\tensor{R}{^3_1_1_3}, \\
    \tensor{R}{^2_3_2_3} &= \tensor{R}{^3_2_3_2} = -\tensor{R}{^2_3_3_2} = -\tensor{R}{^3_2_2_3}.
    \end{split}
\end{equation}
Contrayendo el tensor de Riemann se halla el tensor de Ricci
\begin{equation}
    \tensor{R}{_\alpha_\beta}=\tensor{R}{^\mu_\alpha_\mu_\beta},
\end{equation}
cuyas componentes serán
\begin{subequations}
\begin{align}
    \tensor{R}{_0_0}&=\tensor{R}{^0_0_0_0} + \tensor{R}{^1_0_1_0} + \tensor{R}{^2_0_2_0} + \tensor{R}{^3_0_3_0} \nonumber \\
    &= \frac{2a^\prime-r a^\prime b^\prime +r a^{\prime 2}+r a^{\prime\prime}}{r}e^{-2b} \\
    \tensor{R}{_1_1}&=\tensor{R}{^0_1_0_1} + \tensor{R}{^1_1_1_1} + \tensor{R}{^2_1_2_1} + \tensor{R}{^3_1_3_1} \nonumber \\
    &=\frac{2b^\prime+r a^\prime b^\prime -r a^{\prime 2}-r a^{\prime\prime}}{r}e^{-2b} \\
    \tensor{R}{_2_2}&=\tensor{R}{^0_2_0_2} + \tensor{R}{^1_2_1_2} + \tensor{R}{^2_2_2_2} + \tensor{R}{^3_2_3_2} \nonumber \\
    &= -\frac{a^\prime e^{-2b}}{r}+\frac{b^\prime e^{-2b}}{r}+\frac{1-e^{-2b}}{r} \\
    \tensor{R}{_3_3}&=\tensor{R}{^0_3_0_3} + \tensor{R}{^1_3_1_3} + \tensor{R}{^2_3_2_3} + \tensor{R}{^3_3_3_3} \nonumber \\
    &= -\frac{a^\prime e^{-2b}}{r}+\frac{b^\prime e^{-2b}}{r}+\frac{1-e^{-2b}}{r}.
\end{align}
\end{subequations}
y el escalar de curvatura contrayendo el tensor de Ricci
\begin{align}
    R&=\tensor{\eta}{^\alpha^\beta} \tensor{R}{_\alpha_\beta} \nonumber \\
    &= \tensor{\eta}{^0^0} \tensor{R}{_0_0} + \tensor{\eta}{^1^1} \tensor{R}{_1_1} + \tensor{\eta}{^2^2} \tensor{R}{_2_2} + \tensor{\eta}{^3^3} \tensor{R}{_3_3} \nonumber \\
    &= 2\qty(\frac{b^\prime-a^\prime+r a^\prime b^\prime -r a^{\prime 2}-r a^{\prime\prime}}{r}e^{-2b}+\frac{b^\prime-a^\prime}{r}e^{-2b}+\frac{1-e^{-2b}}{r^2}).
\end{align}
El tensor de Einstein, definido como
\begin{equation}
    \bm{G}=\bm{R}-\frac{1}{2}R\bm{g},
\end{equation}
o en componentes mixtas
\begin{equation}
    \tensor{G}{^\alpha_\beta}=\tensor{R}{^\alpha_\beta}-\frac{1}{2}R\tensor{\delta}{^\alpha_\beta}=\tensor{\eta}{^\alpha^\mu}\tensor{R}{_\mu_\beta}-\frac{1}{2}R\tensor{\delta}{^\alpha_\beta},
\end{equation}
puede ser calculado usando los resultados anteriores
\begingroup
\allowdisplaybreaks
\begin{subequations}
\begin{align}
    \tensor{G}{^0_0} &=  \tensor{\eta}{^0^0}\tensor{R}{_0_0}-\frac{1}{2}R\tensor{\delta}{^0_0} \nonumber \\
    &= - \frac{2a^\prime-r a^\prime b^\prime +r a^{\prime 2}+r a^{\prime\prime}}{r}e^{-2b} -\frac{b^\prime-a^\prime+r a^\prime b^\prime -r a^{\prime 2}+r a^{\prime\prime}}{r}e^{-2b} \nonumber \\
    &\phantom{{}= - \frac{2a^\prime-r a^\prime b^\prime +r a^{\prime 2}+r a^{\prime\prime}}{r}e^{-2b}} -\frac{b^\prime-a^\prime}{r}e^{-2b}+\frac{1-e^{-2b}}{r^2} \nonumber \\
    &= -\frac{1}{r^2}+e^{-2b}\qty(\frac{1}{r^2}-2\frac{b^\prime}{r}), \\
    \tensor{G}{^1_1} &=  \tensor{\eta}{^1^1}\tensor{R}{_1_1}-\frac{1}{2}R\tensor{\delta}{^1_1} \nonumber \\
    &= \frac{2b^\prime+r a^\prime b^\prime -r a^{\prime 2}-r a^{\prime\prime}}{r}e^{-2b} -\frac{b^\prime-a^\prime+r a^\prime b^\prime -r a^{\prime 2}+r a^{\prime\prime}}{r}e^{-2b} \nonumber \\
    &\phantom{{}= - \frac{2a^\prime-r a^\prime b^\prime +r a^{\prime 2}+r a^{\prime\prime}}{r}e^{-2b}}-\frac{b^\prime-a^\prime}{r}e^{-2b}+\frac{1-e^{-2b}}{r^2} \nonumber \\
    &= -\frac{1}{r^2}+e^{-2b}\qty(\frac{1}{r^2}+2\frac{a^\prime}{r}), \\
    \tensor{G}{^2_2}&=\tensor{G}{^3_3} =  \tensor{\eta}{^2^2}\tensor{R}{_2_2}-\frac{1}{2}R\tensor{\delta}{^2_2} \nonumber \\
    &=-\frac{a^\prime e^{-2b}}{r}+\frac{b^\prime e^{-2b}}{r}+\frac{1-e^{-2b}}{r}-\frac{b^\prime-a^\prime+r a^\prime b^\prime -r a^{\prime 2}+r a^{\prime\prime}}{r}e^{-2b} \nonumber \\
    &\phantom{{}= - \frac{a^\prime e^{-2b}}{r}+\frac{b^\prime e^{-2b}}{r}+\frac{1-e^{-2b}}{r}}-\frac{b^\prime-a^\prime}{r}e^{-2b}+\frac{1-e^{-2b}}{r^2} \nonumber \\
    &= e^{-2b}\qty(a^{\prime\prime}-a^{\prime}b^{\prime}+a^{\prime 2}+\frac{a^{\prime}-b^\prime}{r}).
\end{align}
\end{subequations}
\endgroup